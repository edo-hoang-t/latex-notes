\documentclass{article}
\usepackage[utf8]{inputenc}
\usepackage{amsmath}
\usepackage{hyperref} % Important to put at the end because it redefines some Latex commands.
% This package autoly make all the ref clickable and the table of contents clickable and can link to outside webpages via Url() or href() cmds.
\title{Intro to References and Links}
\author{hoang.tran2053018 }
\date{January 2022}

\begin{document}

\maketitle
\section{Hand writing}
This is self-explanatory.

\section{LaTeX}\label{sec:latex-section} % We can use label to make ref to any env.
LaTeX is an amazing software for typesetting. Here is a second-degree polynomial: \[ax^{2}+bx+c\]
If we just use displayed math equation, it's difficult to ref to the equation. Instead, we can put the equation inside \textbf{equation} environment and refer to as normal.
\begin{equation}
\label{eq:2nd_polynomial} % Note for consistency, for equation use _, for section use -
    ax^{2}+bx+c
\end{equation}
You can read more about LaTeX table here: \url{https://www.overleaf.com/learn/latex/Tables#Captions.2C_labels_and_references}. As you see, display the whole URL can be lengthy. Another way here: \href{https://www.overleaf.com/learn/latex/Tables#Captions.2C_labels_and_references}{LaTeX Table - Overleaf}. Both are clickable.

\newpage
\section{MS Word}
Unlike LaTeX discussed in Section \ref{sec:latex-section}, MS Word is more user-friendly. See more in page \pageref{sec:latex-section}. Now I refer to the equation 1 above: see \ref{eq:2nd_polynomial}. Notice we dont have () around 1 here, which is not nice. A way to solve is manually add () besides. Or we can use amsmath package. See \eqref{eq:2nd_polynomial}.

\end{document}

\documentclass{article}
\usepackage[utf8]{inputenc}
\usepackage{amsmath, amsfonts, amsthm}
\usepackage{hyperref} 
\theoremstyle{plain} % default
\newtheorem{theorem}{Theorem}[section]
\theoremstyle{definition}
\newtheorem{definition}[theorem]{Definition} % Notice if we put [theorem] after {Definition} then the effect is different.
\theoremstyle{remark}
\newtheorem*{remark}{Remark}

\title{Theorems, Definitions, Remarks}
\author{hoang.tran2053018 }
\date{January 2022}

\begin{document}

\maketitle

\section{Introduction}
\begin{theorem}[Name of the theorem]
This is a theorem.
\end{theorem}
\begin{proof}
This is a proof.
\end{proof}
\begin{definition}\label{def:1}
This is a definition.
\end{definition}

\begin{remark}
This is a remark.
\end{remark}

\begin{definition}
This is another def.
\end{definition}

Refer to definition \ref{def:1} for more details.
\end{document}
